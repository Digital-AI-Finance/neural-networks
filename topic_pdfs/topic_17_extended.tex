\documentclass[8pt,aspectratio=169]{beamer}
\usetheme{Madrid}
\usepackage{graphicx}
\usepackage{amsmath}
\usepackage{amssymb}
\usepackage{tikz}

% Color definitions
\definecolor{mlblue}{RGB}{0,102,204}
\definecolor{mlpurple}{RGB}{51,51,178}
\definecolor{mllavender}{RGB}{173,173,224}
\definecolor{mllavender2}{RGB}{193,193,232}
\definecolor{mllavender3}{RGB}{204,204,235}
\definecolor{mllavender4}{RGB}{214,214,239}
\definecolor{mlorange}{RGB}{255, 127, 14}
\definecolor{mlgreen}{RGB}{44, 160, 44}
\definecolor{mlred}{RGB}{214, 39, 40}

% Apply custom colors
\setbeamercolor{palette primary}{bg=mllavender3,fg=mlpurple}
\setbeamercolor{palette secondary}{bg=mllavender2,fg=mlpurple}
\setbeamercolor{palette tertiary}{bg=mllavender,fg=white}
\setbeamercolor{palette quaternary}{bg=mlpurple,fg=white}
\setbeamercolor{structure}{fg=mlpurple}
\setbeamercolor{frametitle}{fg=mlpurple,bg=mllavender3}
\setbeamercolor{block title}{bg=mllavender2,fg=mlpurple}
\setbeamercolor{block body}{bg=mllavender4,fg=black}

\setbeamertemplate{navigation symbols}{}
\setbeamersize{text margin left=5mm,text margin right=5mm}

\title{17. Market Prediction Data}
\subtitle{Neural Networks - From Brain to Business}
\date{}

\begin{document}

\begin{frame}[plain]
\titlepage
\end{frame}

\begin{frame}{Learning Goal}
Understand how to prepare financial features for neural network input.
\end{frame}

\begin{frame}{Key Concept (1/3)}
Before training a neural network, we must convert raw market data into numerical features the network can process. This \textbf{feature engineering} step is crucial - the network can only learn patterns present in the features we provide.
\end{frame}

\begin{frame}{Key Concept (2/3)}
For stock prediction, common features include:
- \textbf{Price data}: Returns, moving averages, momentum
- \textbf{Volume}: Trading activity, normalized by average
- \textbf{Sentiment}: News sentiment scores, social media signals
- \textbf{Volatility}: Price variability, option-implied volatility

Each feature should be \textbf{normalized} (scaled to similar ranges) so no single feature dominates due to scale differences. For example, stock prices might be hundreds of dollars while sentiment scores are between 0 and 1.
\end{frame}

\begin{frame}{Key Concept (3/3)}
The \textbf{target variable} is what we're predicting - in binary classification, this might be 1 (price went up) or 0 (price went down).
\end{frame}

\begin{frame}{Visualization}
\begin{center}
\includegraphics[width=0.9\textwidth,height=0.75\textheight,keepaspectratio]{17_market_prediction_data/market_prediction_data.pdf}
\end{center}
\end{frame}

\begin{frame}{Key Formula}
\textbf{Normalization (Min-Max Scaling):}
\[x_{norm} = \frac{x - x_{min}}{x_{max} - x_{min}}\]

\textbf{Normalization (Z-score):}
\[x_{norm} = \frac{x - \mu}{\sigma}\]

Where:
- \textbf{x} = original value
- \(\mathbf{x_min, x_max}\) = minimum and maximum in dataset
- \textbf{mu} = mean
- \textbf{sigma} = standard deviation
\end{frame}

\begin{frame}{Intuitive Explanation}
Think of feature engineering as translation. Raw market data speaks in different "languages" (dollars, shares, percentages). The neural network needs all inputs in the same "language" (normalized numbers near 0).

Without normalization:
- Price: 150.00
- Volume: 1,500,000
- Sentiment: 0.65

The network would focus on volume (biggest numbers) while ignoring sentiment (smallest). After normalization, all features are equally important to start.
\end{frame}

\begin{frame}{Practice Problem 1}
\textbf{Problem 1}

A stock's price history shows minimum = 95, maximum = 105. Today's price is 102. Calculate the min-max normalized value.


\vspace{1em}
\begin{block}{Solution}
\small

\[x_{norm} = \frac{x - x_{min}}{x_{max} - x_{min}} = \frac{102 - 95}{105 - 95} = \frac{7}{10} = 0.70\]

The normalized price is \textbf{0.70}, indicating it's 70\% of the way from minimum to maximum.



\end{block}
\end{frame}

\begin{frame}{Practice Problem 2}
\textbf{Problem 2}

Volume data has mean = 1,000,000 and standard deviation = 250,000. Today's volume is 1,500,000. Calculate the z-score.


\vspace{1em}
\begin{block}{Solution}
\small

\[x_{norm} = \frac{x - \mu}{\sigma} = \frac{1,500,000 - 1,000,000}{250,000} = \frac{500,000}{250,000} = 2.0\]

The z-score is \textbf{2.0}, meaning today's volume is 2 standard deviations above average - unusually high trading activity.



\end{block}
\end{frame}

\begin{frame}{Key Takeaways}
\begin{itemize}
  \item Feature engineering converts raw data to network-friendly format
  \item Normalization ensures all features are on similar scales
  \item Common features: price, volume, sentiment, volatility
  \item Avoid data leakage - only use information available at prediction time
  \item Target variable for classification: 1 (up) or 0 (down)
\end{itemize}
\end{frame}

\end{document}
