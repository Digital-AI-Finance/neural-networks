\documentclass[8pt,aspectratio=169]{beamer}
\usetheme{Madrid}
\usepackage{graphicx}
\usepackage{booktabs}
\usepackage{adjustbox}
\usepackage{multicol}
\usepackage{amsmath}
\usepackage{amssymb}

% Color definitions
\definecolor{mlblue}{RGB}{0,102,204}
\definecolor{mlpurple}{RGB}{51,51,178}
\definecolor{mllavender}{RGB}{173,173,224}
\definecolor{mllavender2}{RGB}{193,193,232}
\definecolor{mllavender3}{RGB}{204,204,235}
\definecolor{mllavender4}{RGB}{214,214,239}
\definecolor{mlorange}{RGB}{255, 127, 14}
\definecolor{mlgreen}{RGB}{44, 160, 44}
\definecolor{mlred}{RGB}{214, 39, 40}
\definecolor{mlgray}{RGB}{127, 127, 127}

% Additional colors for template compatibility
\definecolor{lightgray}{RGB}{240, 240, 240}
\definecolor{midgray}{RGB}{180, 180, 180}

% Apply custom colors to Madrid theme
\setbeamercolor{palette primary}{bg=mllavender3,fg=mlpurple}
\setbeamercolor{palette secondary}{bg=mllavender2,fg=mlpurple}
\setbeamercolor{palette tertiary}{bg=mllavender,fg=white}
\setbeamercolor{palette quaternary}{bg=mlpurple,fg=white}

\setbeamercolor{structure}{fg=mlpurple}
\setbeamercolor{section in toc}{fg=mlpurple}
\setbeamercolor{subsection in toc}{fg=mlblue}
\setbeamercolor{title}{fg=mlpurple}
\setbeamercolor{frametitle}{fg=mlpurple,bg=mllavender3}
\setbeamercolor{block title}{bg=mllavender2,fg=mlpurple}
\setbeamercolor{block body}{bg=mllavender4,fg=black}

% Remove navigation symbols
\setbeamertemplate{navigation symbols}{}

% Clean itemize/enumerate
\setbeamertemplate{itemize items}[circle]
\setbeamertemplate{enumerate items}[default]

% Reduce margins for more content space
\setbeamersize{text margin left=5mm,text margin right=5mm}

% Command for bottom annotation (Madrid-style)
\newcommand{\bottomnote}[1]{%
\vfill
\vspace{-2mm}
\textcolor{mllavender2}{\rule{\textwidth}{0.4pt}}
\vspace{1mm}
\footnotesize
\textbf{#1}
}

\begin{document}

\begin{frame}[plain]
\titlepage
\end{frame}

\begin{frame}{Learning Goal}
Evaluate a neural network trading strategy against a buy-and-hold benchmark.

\bottomnote{This slide establishes the learning objective for this topic}
\end{frame}

\begin{frame}{Key Concept (1/2)}
A \textbf{backtest} simulates how a trading strategy would have performed on historical data. For neural network strategies, we compare:
- \textbf{Buy-and-hold}: Simply buy at start, hold throughout (benchmark)
- \textbf{NN strategy}: Trade based on model predictions (BUY when confidence > threshold)

The backtest reveals whether the model's predictions translate into actual profits. A 70\% accurate model doesn't guarantee outperformance - transaction costs, timing, and magnitude of wins vs losses all matter.

\bottomnote{Understanding this concept is crucial for neural network fundamentals}
\end{frame}

\begin{frame}{Key Concept (2/2)}
Key metrics:
- \textbf{Cumulative returns}: Total profit over the period
- \textbf{Sharpe ratio}: Risk-adjusted returns
- \textbf{Maximum drawdown}: Largest peak-to-trough decline

The ultimate test: Does the strategy beat the benchmark after accounting for realistic trading costs?

\bottomnote{Understanding this concept is crucial for neural network fundamentals}
\end{frame}

\begin{frame}{Visualization}
\begin{center}
\includegraphics[width=0.9\textwidth,height=0.75\textheight,keepaspectratio]{20_trading_backtest/trading_backtest.pdf}
\end{center}

\bottomnote{Visual representations help solidify abstract concepts}
\end{frame}

\begin{frame}{Key Formula}
\textbf{Cumulative return:}
\[R_{total} = \prod_{t=1}^{T} (1 + r_t) - 1\]

\textbf{Sharpe ratio:}
\[\text{Sharpe} = \frac{\bar{r} - r_f}{\sigma_r}\]

Where:
- r-bar = mean return
- r\_f = risk-free rate
- sigma\_r = standard deviation of returns

\textbf{Maximum drawdown:}
\[\text{MDD} = \max_t \left( \frac{\text{Peak}_t - \text{Value}_t}{\text{Peak}_t} \right)\]

\bottomnote{Mathematical formalization provides precision}
\end{frame}

\begin{frame}{Intuitive Explanation}
Imagine two investors:

\textbf{Investor A (Buy-and-hold):} Buys stock on day 1 and holds for a year. No effort, no trading costs, captures all ups and downs.

\textbf{Investor B (NN strategy):} Uses a neural network to decide daily whether to be invested. Aims to be invested on "up" days and out on "down" days.

If Investor B has 70\% accuracy:
- Captures 70\% of "up" days (gains)
- Avoids 70\% of "down" days (losses avoided)

But: Each trade costs money (commissions, spread). And perfect timing is impossible - predictions have some error.

The backtest answers: Does the accuracy advantage overcome the trading costs?

\bottomnote{Intuitive explanations bridge theory and practice}
\end{frame}

\begin{frame}{Practice Problem 1}
\textbf{Problem 1}

Over 100 days: Buy-and-hold returns +15\%. NN strategy has 60\% accuracy with average daily return of +0.2\% on correct UP predictions and -0.15\% on incorrect predictions. Estimate NN strategy returns (ignoring costs).


\vspace{1em}
\begin{block}{Solution}
\small

\textbf{Expected daily return:}

Assume 50\% of days are actually UP:
- 30 days: Predict UP, actually UP (60\% of 50 UP days) -> +0.2\% each
- 20 days: Predict UP, actually DOWN (40\% of 50 DOWN days) -> -0.15\% each
- 20 days: Predict DOWN, actually UP (40\% of 50 UP days) -> 0\% (not invested)
- 30 days: Predict DOWN, actually DOWN (60\% of 50 DOWN days) -> 0\% (not invested)

\textbf{Total return:}
\[R = 30 \times 0.2\% + 20 \times (-0.15\%) + 0 + 0\]
\[R = 6.0\% - 3.0\% = 3.0\%\]

Wait - this is worse than buy-and-hold's 15\%!

\textbf{Why?} With only 60\% accuracy, the strategy misses too many up days (20 missed) and still catches some down days (20 losses). The edge isn't strong enough.

\textbf{Lesson:} Accuracy alone doesn't guarantee profitability.



\end{block}

\bottomnote{Practice problems reinforce understanding}
\end{frame}

\begin{frame}{Practice Problem 2}
\textbf{Problem 2}

Transaction cost is 0.1\% per trade. The NN strategy trades 200 times over the year. How much do costs reduce returns?


\vspace{1em}
\begin{block}{Solution}
\small

\textbf{Transaction cost calculation:}

Total trades: 200
Cost per trade: 0.1\%

\textbf{Total cost:}
\[\text{Costs} = 200 \times 0.1\% = 20\%\]

This is \textbf{20 percentage points} of returns lost to trading costs!

\textbf{Impact:}
- If strategy gross return = 25\%, net return = 25\% - 20\% = 5\%
- If strategy gross return = 15\%, net return = 15\% - 20\% = -5\% (loss!)

\textbf{Lesson:} High-frequency strategies need either very high accuracy or very low costs to be profitable.

\textbf{Solutions:}
- Trade less frequently (e.g., weekly signals)
- Use lower-cost brokers
- Only trade on high-confidence signals (threshold > 0.7)



\end{block}

\bottomnote{Practice problems reinforce understanding}
\end{frame}

\begin{frame}{Key Takeaways}
\begin{itemize}
  \item Backtests simulate strategy performance on historical data
  \item Compare against buy-and-hold benchmark
  \item Transaction costs can eliminate prediction edge
  \item Accuracy doesn't guarantee profitability
  \item Consider risk metrics (drawdown, Sharpe) not just returns
  \item Realistic backtesting includes costs, slippage, and realistic assumptions
\end{itemize}

\bottomnote{These key points summarize the essential learnings}
\end{frame}

\end{document}
