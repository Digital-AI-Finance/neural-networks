\documentclass[8pt,aspectratio=169]{beamer}
\usetheme{Madrid}
\usepackage{graphicx}
\usepackage{amsmath}
\usepackage{amssymb}
\usepackage{tikz}

% Color definitions
\definecolor{mlblue}{RGB}{0,102,204}
\definecolor{mlpurple}{RGB}{51,51,178}
\definecolor{mllavender}{RGB}{173,173,224}
\definecolor{mllavender2}{RGB}{193,193,232}
\definecolor{mllavender3}{RGB}{204,204,235}
\definecolor{mllavender4}{RGB}{214,214,239}
\definecolor{mlorange}{RGB}{255, 127, 14}
\definecolor{mlgreen}{RGB}{44, 160, 44}
\definecolor{mlred}{RGB}{214, 39, 40}

% Apply custom colors
\setbeamercolor{palette primary}{bg=mllavender3,fg=mlpurple}
\setbeamercolor{palette secondary}{bg=mllavender2,fg=mlpurple}
\setbeamercolor{palette tertiary}{bg=mllavender,fg=white}
\setbeamercolor{palette quaternary}{bg=mlpurple,fg=white}
\setbeamercolor{structure}{fg=mlpurple}
\setbeamercolor{frametitle}{fg=mlpurple,bg=mllavender3}
\setbeamercolor{block title}{bg=mllavender2,fg=mlpurple}
\setbeamercolor{block body}{bg=mllavender4,fg=black}

\setbeamertemplate{navigation symbols}{}
\setbeamersize{text margin left=5mm,text margin right=5mm}

\title{04. Neuron Decision Maker}
\subtitle{Neural Networks - From Brain to Business}
\date{}

\begin{document}

\begin{frame}[plain]
\titlepage
\end{frame}

\begin{frame}{Learning Goal}
See how a single neuron makes buy/sell decisions using a threshold.
\end{frame}

\begin{frame}{Key Concept (1/2)}
A single neuron acts as a \textbf{decision maker} by comparing its output to a \textbf{threshold}. For binary classification with sigmoid activation:
- Output > 0.5: Predict Class 1 (BUY)
- Output <= 0.5: Predict Class 0 (SELL)

The neuron computes a weighted combination of inputs, transforms it through the sigmoid function, and produces a probability. The threshold converts this probability into a discrete decision.
\end{frame}

\begin{frame}{Key Concept (2/2)}
In trading terms: if the neuron outputs 0.67 (67\% confidence in price increase), the decision is BUY. If it outputs 0.33 (33\% confidence), the decision is SELL.

The position of the decision boundary in feature space corresponds to where the neuron output equals exactly 0.5 - the point of maximum uncertainty.
\end{frame}

\begin{frame}{Visualization}
\begin{center}
\includegraphics[width=0.9\textwidth,height=0.75\textheight,keepaspectratio]{04_neuron_decision_maker/neuron_decision_maker.pdf}
\end{center}
\end{frame}

\begin{frame}{Key Formula}
\textbf{Neuron computation:}
\[z = w_1 x_1 + w_2 x_2 + b\]
\[\hat{y} = \sigma(z) = \frac{1}{1 + e^{-z}}\]

\textbf{Decision rule:}
\[\text{Decision} = \begin{cases} \text{BUY} & \text{if } \hat{y} > 0.5 \\ \text{SELL} & \text{if } \hat{y} \leq 0.5 \end{cases}\]

\textbf{Boundary condition (y-hat = 0.5):}
\[z = 0 \implies w_1 x_1 + w_2 x_2 + b = 0\]
\end{frame}

\begin{frame}{Intuitive Explanation}
Think of the neuron as a judge weighing evidence:

1. \textbf{Gather evidence}: Each input (price, volume) provides information
2. \textbf{Weight importance}: Some evidence matters more (larger weights)
3. \textbf{Combine and evaluate}: Sum weighted evidence, adjust by bias
4. \textbf{Confidence level}: Sigmoid converts to 0-100\% confidence
5. \textbf{Make decision}: If confidence > 50\%, rule in favor (BUY)

The weights encode what the neuron has learned about which inputs matter and how much.
\end{frame}

\begin{frame}{Practice Problem 1}
\textbf{Problem 1}

A neuron has weights w1 = 0.6 (for price), w2 = 0.4 (for volume), and bias b = -0.3. Given price = 0.8 and volume = 0.5, what is the decision?


\vspace{1em}
\begin{block}{Solution}
\small

\textbf{Step 1: Weighted sum}
\[z = 0.6(0.8) + 0.4(0.5) + (-0.3)\]
\[z = 0.48 + 0.20 - 0.30 = 0.38\]

\textbf{Step 2: Sigmoid activation}
\[\hat{y} = \frac{1}{1 + e^{-0.38}} = \frac{1}{1 + 0.684} = 0.594\]

\textbf{Step 3: Decision}
Since 0.594 > 0.5:
\textbf{Decision: BUY} (59.4\% confidence in price increase)



\end{block}
\end{frame}

\begin{frame}{Practice Problem 2}
\textbf{Problem 2}

Using the same neuron, what values of (price, volume) lie exactly on the decision boundary?


\vspace{1em}
\begin{block}{Solution}
\small

On the boundary, z = 0:
\[0.6 \cdot \text{price} + 0.4 \cdot \text{volume} - 0.3 = 0\]

Solving for volume:
\[\text{volume} = \frac{0.3 - 0.6 \cdot \text{price}}{0.4}\]
\[\text{volume} = 0.75 - 1.5 \cdot \text{price}\]

Example points on the boundary:
- price = 0.0: volume = 0.75
- price = 0.5: volume = 0.0
- price = 0.3: volume = 0.3

These points all yield 50\% confidence (maximum uncertainty).



\end{block}
\end{frame}

\begin{frame}{Key Takeaways}
\begin{itemize}
  \item A neuron outputs a probability via sigmoid activation
  \item Default threshold of 0.5 converts probability to binary decision
  \item The decision boundary is where output = 0.5 (z = 0)
  \item Threshold can be adjusted based on risk tolerance
  \item Single neuron = single linear decision boundary
\end{itemize}
\end{frame}

\end{document}
