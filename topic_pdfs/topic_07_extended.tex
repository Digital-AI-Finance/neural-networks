\documentclass[8pt,aspectratio=169]{beamer}
\usetheme{Madrid}
\usepackage{graphicx}
\usepackage{amsmath}
\usepackage{amssymb}
\usepackage{tikz}

% Color definitions
\definecolor{mlblue}{RGB}{0,102,204}
\definecolor{mlpurple}{RGB}{51,51,178}
\definecolor{mllavender}{RGB}{173,173,224}
\definecolor{mllavender2}{RGB}{193,193,232}
\definecolor{mllavender3}{RGB}{204,204,235}
\definecolor{mllavender4}{RGB}{214,214,239}
\definecolor{mlorange}{RGB}{255, 127, 14}
\definecolor{mlgreen}{RGB}{44, 160, 44}
\definecolor{mlred}{RGB}{214, 39, 40}

% Apply custom colors
\setbeamercolor{palette primary}{bg=mllavender3,fg=mlpurple}
\setbeamercolor{palette secondary}{bg=mllavender2,fg=mlpurple}
\setbeamercolor{palette tertiary}{bg=mllavender,fg=white}
\setbeamercolor{palette quaternary}{bg=mlpurple,fg=white}
\setbeamercolor{structure}{fg=mlpurple}
\setbeamercolor{frametitle}{fg=mlpurple,bg=mllavender3}
\setbeamercolor{block title}{bg=mllavender2,fg=mlpurple}
\setbeamercolor{block body}{bg=mllavender4,fg=black}

\setbeamertemplate{navigation symbols}{}
\setbeamersize{text margin left=5mm,text margin right=5mm}

\title{07. Sigmoid Saturation}
\subtitle{Neural Networks - From Brain to Business}
\date{}

\begin{document}

\begin{frame}[plain]
\titlepage
\end{frame}

\begin{frame}{Learning Goal}
Understand the vanishing gradient problem caused by sigmoid saturation.
\end{frame}

\begin{frame}{Key Concept (1/2)}
The sigmoid function \textbf{saturates} at extreme values. When z is very large (e.g., z > 3) or very small (z < -3), the sigmoid output is nearly 1 or 0, and barely changes regardless of input changes.

This creates the \textbf{vanishing gradient problem}. During backpropagation, gradients flow backward through the network. In saturated regions, gradients become extremely small (near zero), so weights barely update. The network stops learning.
\end{frame}

\begin{frame}{Key Concept (2/2)}
The \textbf{maximum gradient} of sigmoid is only 0.25 (at z = 0). In deep networks, multiplying many small gradients results in effectively zero gradient for early layers - they learn very slowly or not at all.

\textbf{ReLU} solves this by having a constant gradient of 1 for positive inputs. However, ReLU has its own issue: neurons can "die" if z is always negative (gradient = 0).
\end{frame}

\begin{frame}{Visualization}
\begin{center}
\includegraphics[width=0.9\textwidth,height=0.75\textheight,keepaspectratio]{07_sigmoid_saturation/sigmoid_saturation.pdf}
\end{center}
\end{frame}

\begin{frame}{Key Formula}
\textbf{Sigmoid and its derivative:}
\[\sigma(z) = \frac{1}{1 + e^{-z}}\]

\[\sigma'(z) = \sigma(z)(1 - \sigma(z))\]

\textbf{Maximum derivative:}
\[\max(\sigma'(z)) = \sigma(0)(1 - \sigma(0)) = 0.5 \times 0.5 = 0.25\]

\textbf{ReLU and its derivative:}
\[\text{ReLU}(z) = \max(0, z)\]

\[\text{ReLU}'(z) = \begin{cases} 1 & z > 0 \\ 0 & z \leq 0 \end{cases}\]
\end{frame}

\begin{frame}{Intuitive Explanation}
Imagine pushing a ball on different surfaces:
- \textbf{Sigmoid at extremes}: Like pushing a ball on a flat plateau. No matter how hard you push, it barely moves. (Small gradient = little learning)
- \textbf{Sigmoid at center}: Gentle slope, ball rolls smoothly. (Gradient = 0.25)
- \textbf{ReLU positive region}: Constant slope, ball rolls consistently. (Gradient = 1)
- \textbf{ReLU negative region}: Another flat plateau, ball doesn't move.

For efficient learning, we want consistent, non-vanishing gradients throughout.
\end{frame}

\begin{frame}{Practice Problem 1}
\textbf{Problem 1}

Calculate the sigmoid derivative at z = 5. How does it compare to the maximum derivative?


\vspace{1em}
\begin{block}{Solution}
\small

First, calculate sigmoid:
\[\sigma(5) = \frac{1}{1 + e^{-5}} = \frac{1}{1 + 0.0067} = 0.9933\]

Then, calculate derivative:
\[\sigma'(5) = \sigma(5)(1 - \sigma(5)) = 0.9933 \times 0.0067 = 0.0067\]

\textbf{Comparison to maximum:}
\[\frac{0.0067}{0.25} = 0.027 = 2.7\%\]

At z = 5, the gradient is only \textbf{2.7\% of maximum}. Learning is extremely slow in this saturated region.



\end{block}
\end{frame}

\begin{frame}{Practice Problem 2}
\textbf{Problem 2}

A network has 5 layers, each with sigmoid activation. If each layer's gradient is 0.2, what is the total gradient that reaches the first layer?


\vspace{1em}
\begin{block}{Solution}
\small

Gradients multiply through layers (chain rule):
\[\text{Total gradient} = 0.2 \times 0.2 \times 0.2 \times 0.2 \times 0.2 = 0.2^5\]
\[= 0.00032\]

The first layer receives a gradient of only \textbf{0.00032} - practically zero!

With learning rate 0.1:
\[\Delta w = 0.1 \times 0.00032 = 0.000032\]

Weight updates are negligible. The first layer cannot learn effectively.

\textbf{This is the vanishing gradient problem.}



\end{block}
\end{frame}

\begin{frame}{Key Takeaways}
\begin{itemize}
  \item Sigmoid saturates at extreme values (gradient near 0)
  \item Maximum sigmoid gradient is only 0.25
  \item Vanishing gradients prevent learning in deep networks
  \item ReLU has constant gradient = 1 for positive inputs
  \item Modern networks use ReLU (or variants) for hidden layers
\end{itemize}
\end{frame}

\end{document}
