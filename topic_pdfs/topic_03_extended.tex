\documentclass[8pt,aspectratio=169]{beamer}
\usetheme{Madrid}
\usepackage{graphicx}
\usepackage{booktabs}
\usepackage{adjustbox}
\usepackage{multicol}
\usepackage{amsmath}
\usepackage{amssymb}

% Color definitions
\definecolor{mlblue}{RGB}{0,102,204}
\definecolor{mlpurple}{RGB}{51,51,178}
\definecolor{mllavender}{RGB}{173,173,224}
\definecolor{mllavender2}{RGB}{193,193,232}
\definecolor{mllavender3}{RGB}{204,204,235}
\definecolor{mllavender4}{RGB}{214,214,239}
\definecolor{mlorange}{RGB}{255, 127, 14}
\definecolor{mlgreen}{RGB}{44, 160, 44}
\definecolor{mlred}{RGB}{214, 39, 40}
\definecolor{mlgray}{RGB}{127, 127, 127}

% Additional colors for template compatibility
\definecolor{lightgray}{RGB}{240, 240, 240}
\definecolor{midgray}{RGB}{180, 180, 180}

% Apply custom colors to Madrid theme
\setbeamercolor{palette primary}{bg=mllavender3,fg=mlpurple}
\setbeamercolor{palette secondary}{bg=mllavender2,fg=mlpurple}
\setbeamercolor{palette tertiary}{bg=mllavender,fg=white}
\setbeamercolor{palette quaternary}{bg=mlpurple,fg=white}

\setbeamercolor{structure}{fg=mlpurple}
\setbeamercolor{section in toc}{fg=mlpurple}
\setbeamercolor{subsection in toc}{fg=mlblue}
\setbeamercolor{title}{fg=mlpurple}
\setbeamercolor{frametitle}{fg=mlpurple,bg=mllavender3}
\setbeamercolor{block title}{bg=mllavender2,fg=mlpurple}
\setbeamercolor{block body}{bg=mllavender4,fg=black}

% Remove navigation symbols
\setbeamertemplate{navigation symbols}{}

% Clean itemize/enumerate
\setbeamertemplate{itemize items}[circle]
\setbeamertemplate{enumerate items}[default]

% Reduce margins for more content space
\setbeamersize{text margin left=5mm,text margin right=5mm}

% Command for bottom annotation (Madrid-style)
\newcommand{\bottomnote}[1]{%
\vfill
\vspace{-2mm}
\textcolor{mllavender2}{\rule{\textwidth}{0.4pt}}
\vspace{1mm}
\footnotesize
\textbf{#1}
}

\begin{document}

\begin{frame}[plain]
\titlepage
\end{frame}

\begin{frame}{Learning Goal}
Visualize why complex market data cannot be separated by simple linear rules.

\bottomnote{This slide establishes the learning objective for this topic}
\end{frame}

\begin{frame}{Key Concept (1/2)}
When we plot market features against each other (returns vs volume, sentiment vs volatility), we often see \textbf{overlapping clusters} rather than clean separations. Days when the market went up (one class) are scattered throughout the same regions as days when it went down (other class).

\bottomnote{Understanding this concept is crucial for neural network fundamentals}
\end{frame}

\begin{frame}{Key Concept (2/2)}
This overlap explains why simple rules fail. "Buy when volume is high" doesn't work because high volume days include both winners and losers. The relationship between features and outcomes is \textbf{non-linear} - success depends on complex combinations of factors, not single thresholds.

Neural networks excel at finding these complex, non-linear patterns. They can learn decision boundaries that curve through the data, separating classes that no straight line could divide.

\bottomnote{Understanding this concept is crucial for neural network fundamentals}
\end{frame}

\begin{frame}{Visualization}
\begin{center}
\includegraphics[width=0.9\textwidth,height=0.75\textheight,keepaspectratio]{03_problem_visualization/problem_visualization.pdf}
\end{center}

\bottomnote{Visual representations help solidify abstract concepts}
\end{frame}

\begin{frame}{Key Formula}
\textbf{Linear decision boundary attempt:}
\[w_1 \cdot \text{returns} + w_2 \cdot \text{volume} + b = 0\]

This defines a straight line in feature space, but for overlapping data, no such line achieves good separation.

\textbf{Classification error:}
\[\text{Error} = \frac{\text{Misclassified Points}}{\text{Total Points}}\]

\bottomnote{Mathematical formalization provides precision}
\end{frame}

\begin{frame}{Intuitive Explanation}
Imagine sorting marbles by color when red and blue marbles are thoroughly mixed together in a pile. You can't draw a single straight line through the pile to separate them - wherever you draw the line, some marbles of each color end up on the wrong side.

Market data is similar: "up" days and "down" days are interspersed throughout feature space. Simple rules (like "buy when sentiment is positive") misclassify many examples because the relationship is more complex.

\bottomnote{Intuitive explanations bridge theory and practice}
\end{frame}

\begin{frame}{Practice Problem 1}
\textbf{Problem 1}

You plot 100 trading days: 50 "up" days and 50 "down" days. When Volume > 0.6, you find 30 up days and 25 down days. Is "Volume > 0.6 predicts up" a good rule?


\vspace{1em}
\begin{block}{Solution}
\small

Above threshold (Volume > 0.6):
- Up days: 30
- Down days: 25
- Total: 55

Accuracy of "predict UP when Volume > 0.6":
\[\text{Accuracy} = \frac{30}{55} = 54.5\%\]

Below threshold (Volume <= 0.6):
- Up days: 50 - 30 = 20
- Down days: 50 - 25 = 25
- Total: 45

If we predict DOWN below threshold:
\[\text{Accuracy} = \frac{25}{45} = 55.6\%\]

\textbf{Overall accuracy: (30 + 25) / 100 = 55\%}

This is only slightly better than chance (50\%). The rule captures some signal but leaves much unexplained. \textbf{Not a good rule} - the overlap is too significant.



\end{block}

\bottomnote{Practice problems reinforce understanding}
\end{frame}

\begin{frame}{Practice Problem 2}
\textbf{Problem 2}

Why might the combination of volume AND sentiment predict better than either alone?


\vspace{1em}
\begin{block}{Solution}
\small

\textbf{Interaction effects} between features can be more predictive than individual features:

1. \textbf{High volume + positive sentiment}: Strong buying pressure, likely UP

2. \textbf{High volume + negative sentiment}: Strong selling pressure, likely DOWN

3. \textbf{Low volume + positive sentiment}: Weak signal, uncertain

4. \textbf{Low volume + negative sentiment}: Weak signal, uncertain

Neither feature alone captures this pattern:
- High volume alone? Could be buying OR selling
- Positive sentiment alone? Could be acted upon OR ignored

The \textbf{combination} reveals the true signal. This is exactly what neural networks learn - complex interactions between features that simple rules miss.



\end{block}

\bottomnote{Practice problems reinforce understanding}
\end{frame}

\begin{frame}{Key Takeaways}
\begin{itemize}
  \item Market data often shows overlapping classes in feature space
  \item Simple linear rules fail when data is non-linearly separable
  \item Plotting features reveals the complexity of the classification problem
  \item Neural networks can learn non-linear boundaries through overlapping regions
  \item Complete overlap = unpredictable; partial overlap = opportunity for learning
\end{itemize}

\bottomnote{These key points summarize the essential learnings}
\end{frame}

\end{document}
